
\begin{dialog}{半音阶幻想曲,及互格}

\begin{quote}
乌龟在池塘里痛快地涮了涮身子,爬出来甩着水珠。这时有个人从旁边走过,乌龟抬眼一看,原来是阿基里斯。
\end{quote}

\begin{dialogue}

\item[乌龟]嗨,这不是阿基里斯吗?我刚才玩水的时候还想着你呢。

\item[阿基里斯]嘿,真怪了。刚才我在草地上闲逛的时候也想起你来。这个季节草是多么绿啊……

\item[乌龟]是吗?这倒让我记起一件事。我曾有个想法,一直希望能和你共享。你愿意听吗?

\item[阿基里斯]当然,非常愿意。我是说,只要你龟兄别琢磨着把我诱入某个恶意的逻辑圈套,我就愿意听听。

\item[乌龟]恶意的圈套?哎呀,你错怪我了。我哪会有什么恶意?我是个安份的人,从来不打扰别人,成天文绉绉的,吃点草过日子而已。我的脑袋里全是关于事物本性(从我的角度去看)的那些稀奇古怪的东西。我,谦恭的现象观察者,慢吞吞地边走边喷些傻话到空气中去,一点也不惊人。不过,为了让你对我的用意放心,我今天只打算谈谈我的龟壳。你知道,这些事和逻辑没有任何——哪怕是一点点——关系!

\item[阿基里斯]你这么说我就放心了,龟兄。事实上,我倒真有点好奇了。我很想听听你要说些什么,不惊人也没关系。

\item[乌龟]我们来看……我该怎么开头呢?嗯……哎,阿基,你瞧我的壳上什么给你印象最深?

\item[阿基里斯]看上去出奇地干净。

\item[乌龟]谢谢。我刚去游了泳,洗掉了几层上个世纪积存下来的污垢。现在你可以看到我的壳是多么绿了。

\item[阿基里斯]这么健康悦目的绿壳,看着它在阳光下闪闪发亮真让人愉快。

\item[乌龟]绿的?它不是绿的!

\item[阿基里斯]什么?你刚才不是还说你的壳是绿的吗?

\item[乌龟]是说过。

\item[阿基里斯]那好,我们一致了:你的壳是绿的。

\item[乌龟]不,它不是绿的。

\item[阿基里斯]噢,我明白你的把戏了。你在暗示我,你说的不一定是真的;还有,乌龟们玩语言游戏;还有,你的话与事实不一定相符;还有——

\item[乌龟]我肯定没那样。乌龟们神圣地看待言词;乌龟们尊重准确性。

\item[阿基里斯]那你为什么说你的壳是绿的,并且它又不是绿的?

\item[乌龟]我从来没这么说过。不过我倒希望我说了。

\item[阿基里斯]你是希望这么说的了?

\item[乌龟]一点也不。我很遗憾说了那些话,并且完完全全不赞成那些话。

\item[阿基里斯]这无疑和你刚才说的不一致了,你这是自相矛盾!

\item[乌龟]自相矛盾?我从来不自相矛盾。自相矛盾是违反乌龟本性的。

\item[阿基里斯]嗨,这下我可抓住你了,你这滑头。你陷入了一个不折不扣的矛盾,被我抓住了!

\item[乌龟]对,我承认被你抓住了。

\item[阿基里斯]你看你又来了!现在你越来越自相矛盾了!你陷在矛盾之中如此不可自拔,谁也不可能再同你进行辩论了!

\item[乌龟]不见得吧。我同自己辩论没有任何问题,一点也没有。也许问题在于同你辩论。恕我冒昧,我猜你可能是个很矛盾的人,但你被你自己纠缠不清的罗网彻底缠住了,没法看清你自己是多么不一致。

\item[阿基里斯]你这是什么意思?!我一定要让你看到真正矛盾的是你,而不是我。

\item[乌龟]好。如果是这样,你现在的任务就很明确了。还有什么比指出一个矛盾更容易的呢?继续继续——把它指出来。

\item[阿基里斯]嗯……我现在不知道该从哪开始。噢……我知道了。你先是说\pnum{1}:你的壳是绿的,而接着你又说\pnum{2}:你的壳不是绿的。这还有什么可说的?

\item[乌龟]请把矛盾指出来。别旁敲侧击。

\item[阿基里斯]可是——可是——可是……噢,我有点明白了。(我有时候真是太迟钝了!)一定是你和我对于究竟什么是矛盾有不同的看法。麻烦就在这里。好,让我来说清我的意思:如果一个人说一件事,并且同时又否认它,那就是个矛盾。

\item[乌龟]这一招倒挺聪明。我很愿意看看这又能怎么样。大概口技演员是很擅长矛盾的。他们能用嘴同时说一些彼此相反的事情,跟真有那么回事一样。可我不是口技演员。

\item[阿基里斯]我实际上说的是一个人能在一句话里肯定一件事并且又否定它!这不必非要是在同一瞬间里。

\item[乌龟]好吧,不过你并没有给出一个句子。你是给了两个。

\item[阿基里斯]是的——两个互相矛盾的句子!

\item[乌龟]看到你这乱七八糟的思维结构如此暴露无遗,真叫我难过,阿基。你先跟我说矛盾是某种出现在一个单个句子里的东西。然后你又告诉我,你在我所说的两个句子里发现了矛盾。老实说,事情正像我说过的那样,你的思想体系里充满了虚妄,以致于你是在回避看到它有多么不一致。然而,从外面看,这就如同青天白日一样清清楚楚。

\item[阿基里斯]有时候我真被你这种节外生枝的战术搞糊涂了。我简直弄不清我们是在辩论一些无聊至极的东西,还是在辩论某种深刻奥妙的东西!

\item[乌龟]放心好了,乌龟们从不把时间浪费在无聊的东西上。所以,当然是后者。

\item[阿基里斯]我放心了,谢谢。这给了我一个机会好好想想。看来,要让你承认你自相矛盾,必须要有一个逻辑的步骤。

\item[乌龟]很好,很好。我希望那是个简单的步骤,一个不容置疑的步骤。

\item[阿基里斯]正是这样。就是你也会赞成这个步骤。其想法是:由于你相信句子1(“我的壳是绿的”),并且你相信句子2(“我的壳不是绿的”),因此你就会相信把两者结合起来的复合句,不是吗?

\item[乌龟]当然。这倒是挺合理的……只要结合的手段是普遍可接受的。不过我相信在这个问题上我们会是一致的。

\item[阿基里斯]是的。而这样一来我就击败你了!那个复合句就是——

\item[乌龟]但我们在结合句子时必须十分小心。举例来说,你一定承认“人都会死”是个正确的说法,对不对?

\item[阿基里斯]当然。

\item[乌龟]很好。类似地,“饭前洗手”也是对的,是不是?

\item[阿基里斯]谁也不会否认。

\item[乌龟]然后,把它结合起来,我们得到“人都会死于饭前洗手”。但情况并非如此,是吧?

\item[阿基里斯]等等……“人都会死于饭前洗手”?嗯,不对,可是——

\item[乌龟]所以,你也看到了,把两个真句子结合成一个句子并不保险,对吧。

\item[阿基里斯]但是你——你那种结合方法——也太荒唐了!

\item[乌龟]荒唐?你凭什么反对我用这种方法结合句子?难道我非得照你的意愿用些什么其他方法吗?

\item[阿基里斯]你应当用“并且”,而不是用“于”。

\item[乌龟]你是想说,如果你有你的办法,我也应当那样?

\item[阿基里斯]不——是你应当符合逻辑。这与我个人毫无关系。

\item[乌龟]这恰恰就是我没法弄懂你意思的地方。动不动就搬来逻辑以及那些堂而皇之的原则,请你今天别跟我说这些。

\item[阿基里斯]哦,龟兄,别故意让我难受了。你明明知道“并且”这个词的含义所在!用“并且”来结合两个句子是无害的!

\item[乌龟]“无害的”?我的天!简直是胡说八道!明明是个不折不扣的害人诡计,想诱使一个可怜的、走路摇摇晃晃的清白的乌龟陷入致命的矛盾!如果真是那么无害,你为什么还要如此起劲地让我去做?

\item[阿基里斯]我不知道该说什么好了。你让我觉得我自己是个坏人,其实我只有最最纯正的动机。

\item[乌龟]谁都认为自己是这样……

\item[阿基里斯]是我不好——我竟然试图哄骗你,想诱使你陷入自相矛盾。我很难过。

\item[乌龟]你的确该感到惭愧。我知道你想干什么。你算计着让我接受句子3,就是:“我的壳是绿的并且我的壳不是绿的”。这么显眼的谬误是绝不会从一个乌龟的嘴里说出来的!

\item[阿基里斯]哦,太抱歉了,都是我不好。

\item[乌龟]你不必抱歉了。我没觉得不痛快。毕竟,对于我周围那些人的种种不合理做法,我已经习惯了。和你在一起我还是很愉快的,阿基,尽管你的思维缺乏条理。

\item[阿基里斯]是的……可我恐怕已经定型了,在探求真理的征途中,我可能会一错再错。

\item[乌龟]今天的交流对于纠正你的方向是会有些帮助的。再见,阿基。

\item[阿基里斯]再见,龟兄。

\end{dialogue}

\end{dialog}
